%resume.tex template from link at
%http://mboedick.org/code/latex_resume_tips.php
\documentclass[10pt,a4paper]{article}
\usepackage{fullpage}
\pagestyle{empty}
\newcommand{\role}[6]{
\begin{tabular*}{150mm}{l@{\extracolsep{\fill}}r}
#5 & #1--#2 \\ 
\multicolumn{2}{p{145mm}}
{\textbf{#3}#4

#6} 
\end{tabular*}
\vspace{2mm}
 }
\begin{document}
\begin{tabular*}{160mm}{l@{\extracolsep{\fill}}r}
  \textbf{Name} & {\large \textbf{Dr Matthew McDonnell}} \\
  \textbf{Address} & \\
  & \\
  & \\
  \textbf{Phone} & \\
  \textbf{email} & \\
\end{tabular*}
\vspace{0.1in}
\\
\begin{tabular}{p{160mm}}
{\large \textbf{Summary}}\\
\hline
\begin{tabular}{p{145mm}}
Quantitative Analyst focusing on modelling of volatility controlled
multi-asset products. \\
Educated to Doctoral level in quantitative techniques used in experimental
atomic and laser physics.
\\
Expert MATLAB developer with 10+ years experience using MathWorks
tools in academic, technical consulting and quantitative development roles.
\\
Excellent software development skills with knowledge of a range of
languages and software engineering techniques.
\end{tabular}
\end{tabular} \\
\\
\\
\begin{tabular}{p{160mm}}
  {\large \textbf{Employment History}}\\
  \hline
  \role{July 2014}{Present}{Quantitative Analyst}{ in the Solutions Design
  group}{Fidelity Worldwide Investment}
  {Development of volatility controlled multi-asset products 
  within the Fidelity Solutions group. This
  involves modelling and simulation of portfolio management
  strategies and strategic asset allocation choices.}\\
  \role{July 2013}{July 2014}{Senior Quantitative Developer}{}
    {Fidelity Worldwide Investment}
    {}\\
  \role{September 2011}{July 2013}{Quantitative Developer}{}
    {Fidelity Worldwide Investment}
    {Development of analysis and simulation tools for equities investment. This
    covered a number of business areas including quantitative rating of
    securities, portfolio management, trade cost measurement, and
    modelling of investment solutions.}\\
  \role{July 2007}{August 2011}{Technical Consultant}{ in the Consulting
  Services group}{MathWorks}
    {Worked with MathWorks customers in a range of
    industries to increase their productivity and maximize the value of their
    investment in MathWorks tools.  Software 
    development in MATLAB was a major part of this role but
    equally important was providing coaching and integration advice.
    } \\
  \role{September 2006}{April 2007}{Reseach Fellow}{ in the Hydrogen
  Cooling group}{Griffith University}
    {Research into new methods of cooling novel atomic and molecular species
    using ultrafast lasers.  }\\
  \role{July 2003}{August 2006}{Postdoctoral Research Assistant}{ in the
    Ion Trap Quantum Information Processor group}{University of Oxford}{
      Research into experimental implementation of quantum computing.  This
      combined theoretical modelling and development of numerical simulations of
      the system with experimental work. Teaching duties included
      demonstrating on the Physics Computing Course with programming in C,
      Python and Pascal.}\\
  \role{October 2002}{May 2003}{Tutor}{ in First Year Classical Mechanics at
  Keble College}{Keble College, University of Oxford}{}\\
  \role{January 1999}{July 1999}{Research Engineer}{ in the 
      Special Research Centre for Advanced Mineral and Material
      Processing}{University of Western Australia}{}\\
\end{tabular}
\vspace{0.1in}
\begin{tabular}{p{160mm}}
  {\large \textbf{Education}}\\
  \hline
  \role{October 1999}{June 2003} 
     {DPhil in Atomic and Laser Physics.}{ Thesis title: 
     ``Two-Photon Readout Methods for an Ion Trap Quantum Information
     Processor''}
     {University of Oxford}{} \\
  \role{February 1993}{November 1998}
      {BSc (Chemical Physics) (hons. 1st Class),
       BE (Materials) (hons. 1st Class)}{}
      {University of Western Australia}{}
    \\
    \begin{tabular*}{150mm}{ll}
      \textbf{Prizes:}\\
        1996: & Faculty of Science Medal for best Honours Science
        Student\\
        1996: & J.A. Wood Memorial Prize for best Honours Student in\\
        & the Faculties of Science, Engineering, Medicine, Agriculture
        and Dentistry\\
         1999: & Awarded a Commonwealth Scholarship to study for\\
         & a DPhil at the University of Oxford
    \end{tabular*}

%   \item
%     \begin{tabular*}{150mm}{l@{\extracolsep{\fill}}r}
%       February 1988--November 1992 & Duncraig Seniour High School\\
%       TEE Subjects:  Applicable Mathematics, Calculus, &\\
%       Chemistry, Computing, Physics, English Literature &
%     \end{tabular*}
\end{tabular} \\
%
\begin{tabular}{p{160mm}}
{\large \textbf{Publications}}\\
\hline
\begin{tabular}{p{145mm}}
%\begin{tabular*}{150mm}{l}
\begin{itemize}
  \item ``Memory coherence of a sympathetically cooled trapped-ion
    qubit'', Home JP, McDonnell MJ, Szwer DJ, Keitch BC, Lucas DM, Stacey DN,
    Steane AM, \textit{Phys. Rev. A} \textbf{79} 050305 (2009)
  \item ``Long-lived mesoscopic entanglement outside the Lamb-Dicke
    regime'', McDonnell MJ, Home JP, Lucas DM, Imreh G, Keitch BC, Szwer DJ,
    Thomas NR, Webster SC, Stacey DN, Steane AM, \textit{Phys. Rev. Lett.}
    \textbf{98} 063603 (2007)
  \item ``Deterministic entanglement and tomography of ion spin qubits'',
    Home JP, McDonnell MJ, Lucas DM, Imreh G, Keitch BC, Szwer DJ, Thomas NR,
    Webster SC, Stacey DN, Steane AM, \textit{New J. Phys.} \textbf{8} (2006)
  \item ``Laser linewidth effects in quantum state discrimination by
    electromagnetically induced transparency'',  McDonnell MJ,
    Stacey DN, and Steane AM, \textit{Phys. Rev. A} \textbf{70}
    053802 (2004) 
  \item ``High-efficiency detection of a single quantum of angular
    momentum by suppression of optical pumping'',  McDonnell MJ,
    Stacey JP, Webster SC, Home JP, Ramos A, Lucas DM, Stacey DN,
    Steane AM, 
    \textit{Phys. Rev. Lett.} \textbf{93} 153601 (2004)
  \item ``Isotope-selective photoionization for calcium ion
    trapping'', Lucas DM, Ramos A, Home JP, McDonnell MJ, Nakayama S,
    Stacey JP, Webster SC, Stacey DN, Steane AM,
    \textit{Phys. Rev. A} \textbf{69} 012711 (2004)
  \item ``Oxford Ion Trap Quantum Computing Project'', Lucas DM, Donald CJS,
    Home JP, McDonnell M, Ramos A, Stacey DN, Stacey JP, Steane AM,
    Webster SC, \textit{Phil. Tans. R. Soc. Lond. A} \textbf{361} 1401
    (2003)
  \item ``Search for correlation effects in linear chains of trapped
    Ca$^{+}$ ions'', Donald CJS, Lucas DM, Barton PA, McDonnell MJ,
    Stacey JP, Stevens DA, Stacey DN, Steane AM,
    \textit{Europhys. Lett.} \textbf{51} 388--394 (2000)
  \item ``Magnetisation of thin films under oblique field
    conditions'', McDonnell M, Street R, Woodward RC, Chapman JN,
    \textit{J. Magnetism and Magnetic Materials} \textbf{177}
    1281--1282 (1998)
  \item ``Photocurrent autocorrelation of femtosecond laser pulses in
    poly(p-phenylene vinylene)'', Samoc M, Samoc A, Luther-Davies B,
    Dowd A, McDonnell M, \textit{J. Phys. D} \textbf{30} 895--899 (1997)
\end{itemize}
\end{tabular}
\end{tabular}
\\
{\large \textbf{Research Interests}}
\begin{itemize}
\item For the last 5 years I have been involved in an ion trap quantum
  information experiment at the University of Oxford.  The goal of
  this research has been to control and measure the quantum behaviour
  of individual ions of calcium trapped in an ion trap, in order to
  construct the fundamental building blocks of a quantum information
  processing system.  The experimental work has involved laser
  cooling, computer control of experiments and the design and
  construction of optical systems.  The theoretical side of the work
  has involved simulation of the atomic and optical systems using
  programming languages such as \textit{Matlab} and \textit{Mathematica}. 
\item I am interested in applying the techniques I have learnt over
  the course of my career to interdisciplinary areas of science 
  such as nanotechnology and biophysical systems.
\end{itemize}

{\large \textbf{Skills}}
\begin{itemize}
  \item Experimental atomic physics, in particular frequency
    stabilisation of infrared and ultraviolet diode lasers, and  
    ion trap experiments. 
  \item Experience with a wide range of experimental techniques including
    TEM, SEM, AFM, Kerr microscopy, materials testing and analytical chemistry
    techniques.
  \item Numerical simulation of the evolution of atomic systems using
    \textit{Matlab}.
   \item Teaching experience ranging from laboratory demonstrating for groups
    of 10-20 students to Oxford tutorial teaching for groups of 
    2-7 students. 
  \item Good computer skills. Proficient in the use of Windows
    (95/98/2000/NT/XP) and UNIX (Solaris, Linux, FreeBSD) operating
    systems.
  \item Document preparation: \LaTeX, Xfig, Word, Excel, Powerpoint,
    HTML.
  \item Programming languages:  \textit{Matlab}, \textit{Mathematica},
    Ocaml, C, \textit{Labview}, Pascal, Perl, Java.
\end{itemize}

% {\large \textbf{References}}
% %(Available on request)
\end{document}
